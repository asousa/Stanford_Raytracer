\ifx\pdfoutput\undefined % We're not running pdftex
\documentclass[10pt]{article}
\else
\documentclass[pdftex,10pt]{article}
\fi

\usepackage{newcent}
\usepackage[OT1,T1]{fontenc} 

\ifx\pdfoutput\undefined % We're not running pdftex
\usepackage[dvips]{graphicx,color}
\else
\usepackage[pdftex]{graphicx,color}
\fi


\usepackage{placeins}
\usepackage{cite}      % Written by Donald Arseneau
\usepackage{fancyhdr}
%\usepackage{colortab}
\usepackage{array}
%\usepackage{pstricks}
% interferes with longtable. grack!
%\usepackage{colortbl}
%\usepackage{longtable}

%\arrayrulecolor{green}
%\doublerulesepcolor{green}

\ifx\pdfoutput\undefined % We're not running pdftex
\else
\usepackage[colorlinks,hyperindex,linkcolor={blue},citecolor={blue},urlcolor={re
d}]{hyperref}
% supress the ugly colored boxes around the active parts of the
% text.
\def\pdfBorderAttrs{/Border [0 0 0] } % No border around Links
% Some further Color tuning. Use xcolorsel for help with the colors
%\definecolor{links}{rgb}{0.2116,0.0104,0.7716} % BlueViolet
\definecolor{links}{rgb}{0.0,0.0,0.0} % BlueViolet
\def\LinkColor{links}
\definecolor{black}{rgb}{0,0,0}
\def\AnchorColor{anchors}
\fi

\definecolor{darkgray}{rgb}{0.3,0.3,0.3}
%\renewcommand{\familydefault}{\sfdefault}
\setlength{\oddsidemargin}{0.0in}
\setlength{\evensidemargin}{0.0in}
\setlength{\textwidth}{6.5in}
\setlength{\headheight}{0.2in}
\setlength{\topmargin}{-0.3in}
\setlength{\textheight}{8.9in}
\setlength{\topskip}{0.3in}
\setlength{\parindent}{0in}
\setlength{\parskip}{10pt}

\renewcommand\floatpagefraction{1.00}
\renewcommand\topfraction{.85}
\renewcommand\bottomfraction{.85}
\renewcommand\textfraction{.15}   
\setcounter{totalnumber}{50}
\setcounter{topnumber}{50}
\setcounter{bottomnumber}{50}

\hyphenation{FPGA FPGAs}

\newenvironment{codelist}{\begin{list}{}{
    \setlength{\labelwidth}{1.3in}
    \setlength{\leftmargin}{1.5in}
    \setlength{\itemindent}{0in}
    \setlength{\itemsep}{0.0ex}
    \setlength{\parsep}{2.50ex}
    \setlength{\labelsep}{.15in}}}
    {\end{list}}

\newenvironment{codelist2}{\begin{list}{}{
    \setlength{\labelwidth}{0in}
    \setlength{\leftmargin}{.25in}
    \setlength{\itemindent}{0in}
    \setlength{\itemsep}{0.0ex}
    \setlength{\rightmargin}{0in}
    \setlength{\parsep}{1.5ex}
    \setlength{\labelsep}{.25in}}}
    {\end{list}}

\ifx\pdfoutput\undefined % We're not running pdftex
\else
% Don't let graphicx grab png files (they're not what we want, we want
% the pdf versions with the annotations when building a pdf file).
\DeclareGraphicsExtensions{.pdf}
\fi
\ifx\pdfoutput\undefined % We're not running pdftex
\else
\fi

\begin{document}

\title{Raytracer}
\section{Directory structure}
{\bf bin} - executables and matlab driver scripts \\
{\bf fortran} - Fortran raytracer source code \\
{\bf gcpm} - GCPM (a plasmasphere model) \\
{\bf iri2007} - IRI (an ionosphere model, required by GCPM) \\
{\bf lapack-3.2.1} - Lapack and BLAS, used for solving Ax=b \\
{\bf manual} - Documentation \\
{\bf matlab} - Matlab raytracer \\
{\bf othersrc} - Original distribution files for 3rd party sources \\
{\bf tricubic-for} - Tricubic interpolator \\
{\bf tsyganenko} - Tsyganenko magnetosphere model \\
{\bf xform} - coordinate transformation tools (single precision) \\
{\bf xform\_double} - coordinate transformation tools (double
precision) \\

\section{Sample driver script}
The bin/ directory contains a Matlab script ``testrun.m''.  This sets
up a few sample raytracer runs.  It also illustrates the use and
parsing of the basic tools to extract model parameters and plot them.
Basic operation:

{\bf Set up global parameters}
\begin{itemize}
\item {\bf w} - Ray frequency in rad/s
\item {\bf dt0} - Initial timestep in seconds
\item {\bf dtmax} - Maximum allowable timestep in seconds
\item {\bf root} - Which root of the appleton-hartree equation
  (2=whistler in the magnetosphere)
\item {\bf fixedstep} - 
  Whether to use fixed steps (1) or adaptive timesteps (0).
  Fixed steps have NO way to recover if you pop outside the resonance
  cone, so adaptive is always recommended
\item {\bf maxerr} - Error bound for adaptive timestepping
\item {\bf maxsteps} - Maximum number of steps
\item {\bf modelnum} - Model (1=ngo, 2=GCPM (SLOW!!), 3=interpolated)
\end{itemize}

{\bf Choose a plasmasphere model} ({\bf modelnum})
\begin{itemize}
\item {\bf 1}: Ngo model, the classic radial diffusion model used
  by the original raytracer.  This requires the same input file
  (e.g. ``newray.in'') as used by the original raytracer to set
  up the plasmasphere parameters (ducts, knee distance, etc.).
  newray.in here is ONLY used to set up the plasmasphere
  parameters for this model.  The new raytracer uses a different
  format for inputting rays.
\item {\bf 2}: GCPM model.  This is a complete plasmasphere model
  but is slow in practice.  It takes as input a Kp index (see below).
\item {\bf 3}: Interpolated model.  This takes as input a gridded set
  of number densities (format: log base e of the number density in
  $m^{-3}$.  E.g., use \\``gcpm\_dens\_model\_buildgrid'' to generate
  a grid for use with this model.
\end{itemize}

{\bf Set up plasmasphere model parameters}
\begin{itemize}
  \item{\bf Model 1: Ngo Parameters}
    \begin{itemize}
    \item{\bf ngo\_configfile}:
      newray input filename
    \item{\bf yearday}:
      year and day, e.g., 1999098
    \item{\bf milliseconds\_day}:
      milliseconds of day
    \item{\bf use\_tsyganenko}:
      use Tsyganenko magnetic field model? (1=use, 0=do not use)
    \item{\bf use\_igrf}:
      use IGRF magnetic field model or dipole model? (1=IGRF, 0=dipole)
    \item{\bf tsyganenko\_Pdyn}:
      Dynamic solar wind pressure in nP, used by the Tsyganenko model
      - 0.5 and 10 nPa
    \item{\bf tsyganenko\_Dst}:
      Dst in nT, used by the Tsyganenko model - between -100 and +20 in nT
    \item{\bf tsyganenko\_ByIMF}:
      IMF y component in nT, used by the Tsyganenko model - between
      -10 and +10 nT -
    \item{\bf tsyganenko\_BzIMF}:
      IMF z component in nT, used by the Tsyganenko model - between
      -10 and +10 nT
    \end{itemize}
  \item{\bf Model 2:  GCPM Parameters}
    \begin{itemize}
    \item{\bf gcpm\_kp}:
      kp index
    \item{\bf yearday}:
      year and day, e.g., 1999098
    \item{\bf milliseconds\_day}:
      milliseconds of day
    \item{\bf use\_tsyganenko}:
      use Tsyganenko magnetic field model? (1=use, 0=do not use)
    \item{\bf use\_igrf}:
      use IGRF magnetic field model or dipole model? (1=IGRF, 0=dipole)
    \item{\bf tsyganenko\_Pdyn}:
      Dynamic solar wind pressure in nP, used by the Tsyganenko model
      - 0.5 and 10 nPa
    \item{\bf tsyganenko\_Dst}:
      Dst in nT, used by the Tsyganenko model - between -100 and +20 in nT
    \item{\bf tsyganenko\_ByIMF}:
      IMF y component in nT, used by the Tsyganenko model - between
      -10 and +10 nT -
    \item{\bf tsyganenko\_BzIMF}:
      IMF z component in nT, used by the Tsyganenko model - between
      -10 and +10 nT
    \end{itemize}
  \item{\bf Model 3: Interpolated parameters}
    \begin{itemize}
    \item{\bf interp\_interpfile}:
      grid filename
    \item{\bf yearday}:
      year and day, e.g., 1999098
    \item{\bf milliseconds\_day}:
      milliseconds of day
    \item{\bf use\_tsyganenko}:
      use Tsyganenko magnetic field model? (1=use, 0=do not use)
    \item{\bf use\_igrf}:
      use IGRF magnetic field model or dipole model? (1=IGRF, 0=dipole)
    \item{\bf tsyganenko\_Pdyn}:
      Dynamic solar wind pressure in nP, used by the Tsyganenko model
      - 0.5 and 10 nPa
    \item{\bf tsyganenko\_Dst}:
      Dst in nT, used by the Tsyganenko model - between -100 and +20 in nT
    \item{\bf tsyganenko\_ByIMF}:
      IMF y component in nT, used by the Tsyganenko model - between
      -10 and +10 nT -
    \item{\bf tsyganenko\_BzIMF}:
      IMF z component in nT, used by the Tsyganenko model - between
      -10 and +10 nT
    \end{itemize}

  \item{\bf Model 4: Scattered interpolator parameters}
    \begin{itemize}
    \item{\bf interp\_interpfile}:
      grid filename
    \item{\bf yearday}:
      year and day, e.g., 1999098
    \item{\bf milliseconds\_day}:
      milliseconds of day
    \item{\bf use\_tsyganenko}:
      use Tsyganenko magnetic field model? (1=use, 0=do not use)
    \item{\bf use\_igrf}:
      use IGRF magnetic field model or dipole model? (1=IGRF, 0=dipole)
    \item{\bf tsyganenko\_Pdyn}:
      Dynamic solar wind pressure in nP, used by the Tsyganenko model
      - 0.5 and 10 nPa
    \item{\bf tsyganenko\_Dst}:
      Dst in nT, used by the Tsyganenko model - between -100 and +20 in nT
    \item{\bf tsyganenko\_ByIMF}:
      IMF y component in nT, used by the Tsyganenko model - between
      -10 and +10 nT -
    \item{\bf tsyganenko\_BzIMF}:
      IMF z component in nT, used by the Tsyganenko model - between
      -10 and +10 nT
    \item{\bf scattered\_interp\_window\_scale}: This is the factor
      above the maximum sample spacing of the moving least squares
      sphere (the points that are actually considered for
      interpolation).  Above 1.1 or so is usually safe, but set to 1.5
      or even larger for robustness.
    \item{\bf scattered\_interp\_order}:
      The order of the monomials used for interpolation (2 is usually fine).
    \item{\bf scattered\_interp\_exact}:
      Whether to use (1) or not (0) exact interpolation.  Using
      inexact (0) often generates smoother and more accurate results
      around discontinuities or high gradients, at the expense of not 
      being an exact interpolant.
    \item{\bf scattered\_interp\_local\_window\_scale}:
      During evaluation, the minimum average spacing in a neighborhood
      around the interpolation point is used to shrink or grow the
      weighting window.  This gives better results on strongly
      inhomogeneous grids.  Setting to some small value like 2.0 will
      resolve sharp features better.  Setting very large, like 10,
      will smooth out sharp features but will generally be smoother
      and suffer from fewer artifacts.
    \end{itemize}
\end{itemize}

\section{Building}

In the top-level directory, there is a makefile.  Run ``make clean''
then ``make'' in that directory to rebuild everything.  Compilation
requires a Fortran 90 compiler (possibly with some fortran 95
features).  It is tested with g95 and gfortran.  Some of the
third-party packages require static variables; in g95 this is
``-fstatic''; in gfortran, ``-fno-automatic''.  Check and modify the
Makefiles accordingly.

Under Windows, the easiest way to compile this package is to download
MinGW and the MinGW g95 package.  After installing, modify your PATH
and then run ``make clean'' then ``make''.

\section{Tools}

{\bf dumpmodel}

This tool dumps a plasmasphere/magnetic field model to an ASCII text
file.  Run the tool without arguments to see usage.  A typical run
would be:
\begin{verbatim}
 dumpmodel --minx=0 --maxx=3.1855e+007 --miny=0 --maxy=0 \
   --minz=-1.59275e+007 --maxz=1.59275e+007 --nx=200 --ny=1 \
   --nz=200 --filename=dumpout.txt --modelnum=1 --yearday=2001001 \
   --milliseconds_day=0 --use_tsyganenko=0 --use_igrf=0 \
   --tsyganenko_Pdyn=4 --tsyganenko_Dst=0 --tsyganenko_ByIMF=0 \
   --tsyganenko_BzIMF=-5 --ngo_configfile=newray.in
\end{verbatim}
This will output a textfile named ``dumpout.txt''.  The first line of
the file is a header:
\begin{verbatim}
nspec nx ny nz
\end{verbatim}
In order: the number of species, number of
samples in the x direction, number of samples in the y direction,
and number of samples in the z direction, respectively.  The second
line is:
\begin{verbatim}
minx maxx miny maxy minz maxz
\end{verbatim}
In order: minimum extent in the x direction, maximum extent in the x
direction, minimum extent in the y direction, maximum extent in the y
direction, minimum extent in the z direction, and maximum extent in
the z direction.

The remainder of the file is the data.

The Matlab function {\bf readdump.m} reads the output from this tool.

{\bf gcpm\_dens\_model\_buildgrid}

This tool is similar to the model dump tool, but only outputs the
plasmasphere parameters and optionally their derivatives, for use by
Model 3, the interpolated model.  GCPM is prohibitively slow so for a
large number of runs, it is more efficient to generate a large grid
once over the region of interest, and then on later runs, interpolate
over this grid.  Note that the format of this file is in log base e of
the electron number density in $m^{-3}$!

Run the program without arguments to get the usage.  A typical run
would be:
\begin{verbatim}
gcpm_dens_model_buildgrid --minx-1e7 --maxx=1e7 --miny=0 --maxy=0 \
  --minz=-1e7 --maxz=1e7 --nx=40 --ny=1 --nz=40 --compder=0 \
  --filename=out.txt --gcpm_kp=4.0 --yearday=2001001 --milliseconds_day=0
\end{verbatim}

One file has already been generated for testing:

{\bf gcpm\_kp4\_2001001\_L10\_80x80x80\_noderiv.txt} - kp=4,
yearday=2001001, range=+-10 earth radii in every direction,
80x80x80 cells, no derivatives calculated at the faces.

The Matlab function {\bf readinterpolationgrid.m} reads the output
from this tool.  

{\bf gcpm\_dens\_model\_buildgrid\_random}

This tool uses stratified random sampling to sample GCPM on a random
grid for use with Model 4, the scattered interpolated model.  GCPM is
prohibitively slow so for a large number of runs, it is more efficient
to generate a large grid once over the region of interest, and then on
later runs, interpolate over this grid.  Note that the format of this
file is in log base e of the electron number density in $m^{-3}$!

Run the program without arguments to get the usage.  A typical run
would be:
\begin{verbatim}
  gcpm_dens_model_buildgrid_random --minx=-6.371e7 --maxx=6.371e7 \
    --miny=-6.371e7 --maxy=6.371e7 --minz=-6.371e7 --maxz=6.371e7 \
    --n_zero_altitude=5000 --n_iri_pad=20000 --n_initial_radial=0 \
    --n_initial_uniform=200000 --initial_tol=1.0 --max_recursion=80 \
    --adaptive_nmax=600000 --filename=out.txt --gcpm_kp=4.0 \
    --yearday=2001001 --milliseconds_day=0
\end{verbatim}

One file has already been generated for testing:

{\bf gcpm\_kp4\_2001001\_L10\_random\_5000\_20000\_0\_200000\_600000.txt} - kp=4,
yearday=2001001, range=+-10 earth radii in every direction,
about 825000 total samples, 20000 extra samples in the ionopshere,
5000 extra samples at the earth's surface, 200000 uniform random
samples, and 600000 samples used for refinement.

{\bf Explanation of parameters:}
\begin{itemize}
\item {\bf n\_zero\_altitude}

  Number of initial sample points at zero altitude.  This number is for the
  WHOLE earth.
\item {\bf  n\_iri\_pad      }

  Number of initial points to sample at IRI altitudes (up to 2000 km).
  The gradients here are strong so lots of sampling is required to
  keep the errors low.  This number is for the WHOLE earth.
\item {\bf n\_initial\_radial}

  Number of points to sample everywhere else on radially-weighted random points
\item {\bf n\_initial\_uniform}

 Number of points to sample everywhere else, uniform distribution over
 the bounds given.
\item {\bf initial\_tol     }

  Adaptive oversampling starting tolerance (1.0 is suitable for GCPM).
\item  {\bf max\_recursion   }

  Adaptive oversampling recursion depth (set to around 80 just to be safe).
\item {\bf adaptive\_nmax   }

  Maximum number of adaptive samples that are allowed.  The tolerance
  will be successively halved until this number of samples is
  exceeded.
\item {\bf filename        }

  output filename
\item {\bf inputfile       }

  (optional) existing output file to read in any points in this file
  will be made part of the initial grid and will also get pushed to
  the output.
\item {\bf gcpm\_kp         }

  kp index (GCPM parameter)
\item {\bf yearday         }

  year and day, e.g., 1999098 (GCPM parameter)
\item {\bf milliseconds\_day}

  milliseconds of day (GCPM parameter)
\end{itemize}

\section{Raytracer}

The core raytracer tool is called ``raytracer''.  The input arguments
are identical to those described in testrun.m.  The command-line
arguments are as follows:

\begin{verbatim}
 Usage:
   program --param1=value1 --param2=value2 ...
   
 --w             frequency in rad/s
 --dt0           initial timestep in seconds
 --dtmax         maximum timestep in seconds
 --tmax          maximum time for simulation
 --root          root number (2=whistler at VLF frequencies)
 --fixedstep     fixed timesteps (1) or adaptive (0)
 --maxerr        maximum error for adaptive timestepping
 --maxsteps      maximum number of timesteps
 --inputraysfile ray input filename
 --outputfile    output filename
 --modelnum      (1) Ngo model
                 (2) GCPM ionosphere model
                 (3) Interpolated model
  Ngo Parameters (required if model 1 is chosen):
    --ngo_configfile     newray input filename
    --yearday            year and day, e.g., 1999098
    --milliseconds_day   milliseconds of day
    --use_tsyganenko     (1=use, 0=do not use)
    --use_igrf           (1=use, 0=do not use)
    --tsyganenko_Pdyn    between 0.5 and 10 nPa
    --tsyganenko_Dst     between -100 and +20 in nT
    --tsyganenko_ByIMF   between -10 and +10 nT
    --tsyganenko_BzIMF   between -10 and +10 nT
  GCPM Parameters (required if model 2 is chosen):
    --gcpm_kp            kp index
    --yearday            year and day, e.g., 1999098
    --milliseconds_day   milliseconds of day
    --use_tsyganenko     (1=use, 0=do not use)
    --use_igrf           (1=use, 0=do not use)
    --tsyganenko_Pdyn    between 0.5 and 10 nPa
    --tsyganenko_Dst     between -100 and +20 in nT
    --tsyganenko_ByIMF   between -10 and +10 nT
    --tsyganenko_BzIMF   between -10 and +10 nT
  Interp parameters (required if model 3 is chosen):
    --interp_interpfile  grid filename
    --yearday            year and day, e.g., 1999098
    --milliseconds_day   milliseconds of day
    --use_tsyganenko     (1=use, 0=do not use)
    --use_igrf           (1=use, 0=do not use)
    --tsyganenko_Pdyn    between 0.5 and 10 nPa
    --tsyganenko_Dst     between -100 and +20 in nT
    --tsyganenko_ByIMF   between -10 and +10 nT
    --tsyganenko_BzIMF   between -10 and +10 nT
  Scattered interp parameters (required if model 4 is chosen):
    --interp_interpfile  data filename
    --yearday            year and day, e.g., 1999098
    --milliseconds_day   milliseconds of day
    --use_tsyganenko     (1=use, 0=do not use)
    --use_igrf           (1=use, 0=do not use)
    --tsyganenko_Pdyn    between 0.5 and 10 nPa
    --tsyganenko_Dst     between -100 and +20 in nT
    --tsyganenko_ByIMF   between -10 and +10 nT
    --tsyganenko_BzIMF   between -10 and +10 nT
    --scattered_interp_window_scale
                         window radius scale factor above
                         maximum sample spacing
    --scattered_interp_order
                         monomial order
    --scattered_interp_exact
                         exact(1) or inexact(0)
    --scattered_interp_local_window_scale
                         scale factor above minimum average
                         sample spacing
\end{verbatim}

The input file ``inputraysfile'' gives the ray starting positions and
initial wavenormal directions, one ray per line, e.g.:
\begin{verbatim}
  1.0E007 0.0 0.0 0.0 0.0 1.0
\end{verbatim}
This input file launches one ray at position (1e7,0,0) with an initial
wavenormal direction of (0,0,1).  By convention, every model uses SM
(solar magnetic) coordinates.

The output file ``outputfile'' records the ray positions, the time
since start, positions, wavenormals, relative group velocities, and
the magnetic field and plasma parameters along the ray path.  The
Matlab script ``readrayoutput.m'' parses this file.

\section{Adapting a new model}
The raytracer includes 3 default plasmasphere models (Ngo, GCPM,
interpolated), two magnetic field models (dipole and IGRF), and one
magnetic field correction model (Tsyganenko).  The raytracer uses
these models to compute the dispersion relation at arbitrary points in
space.

Adding a new model requires adding a new {\em adapter} to the
raytracer and recompiling.  A sample adapter is included in
``skeleton\_dens\_model\_adapter.f95''.  The adapter, reduced to its
bare minimum, is simply a Fortran 90 module that contains a single
subroutine that returns the plasma parameters for a spatial position,
and conforms to a specific interface:

\begin{verbatim}
  ! Implementation of the plasma parameters function.
  ! Inputs:
  !   x - position vector in cartesian (SM) coordinates
  ! Outputs:
  !  qs - vector of species charges
  !  Ns - vector of species densities in m^-3
  !  ms - vector of species masses in kg
  ! nus - vector of species collisions in s^-1
  !  B0 - cartesian (SM) background magnetic field in Tesla
  ! In/out:
  ! funcPlasmaParamsData - arbitrary callback data 
  subroutine funcPlasmaParams(x, qs, Ns, ms, nus, B0, funcPlasmaParamsData)
\end{verbatim}

This function, which you should implement, takes as input a 3-vector x
(the position in SM coordinates in meters), optionally some arbitrary
state data in the character array ``funcPlasmaParamsData'', and
outputs the quantities qs, Ns, ms, nus, and B0 for that position (this
function should also allocate space for qs, Ns, ms, and nus if they
are not already allocated).  qs, Ns, ms, and nus should all be
identically-sized vectors, one for each plasma species, and return the
charges ``qs'' in C, number densities ``Ns'' in $m^{-3}$, masses
``ms'' in kg, and collision frequency (currently unused!) in $s^{-1}$.
B0 is a simple cartesian 3-vector giving the background magnetic field
in Teslas.

Most 3rd-party models require some additional state variables besides
position.  These can be set in module-level variables (essentially
scoped global variables), or, better, in a callback state variable
``funcPlasmaParamsData''.  ``funcPlasmaParamsdata'' is a simple
allocatable character array.  The Fortran intrinsic function TRANSFER
can be used to transfer any derived type into and out of this state
variable, fulfilling the same role held by a ``void *'' argument in C.

For example, suppose we have one state variable ``use\_igrf'' that we
wish to set before starting the raytracer.
``skeleton\_dens\_model\_adapter'' illustrates.  First we define a
derived type as a container:
\begin{verbatim}
  type :: StateData
     ! Whether to use (1) IGRF or not use (0) and use dipole instead
     integer*4 :: use_igrf
  end type StateData
\end{verbatim}

We then define another derived type containing a pointer to this
derived type.  This is the data we will actually pass through the
``funcPlasmaParamsData'' parameter.  Note that the pointer must be
contained within a derived type so TRANSFER knows how to handle it
properly:
\begin{verbatim}
  ! Pointer container type.  This is the data that is actually marshalled.
  type :: StateDataP 
     type(StateData), pointer :: p
  end type StateDataP
\end{verbatim}

Then, in the main fortran file ``raytracer\_driver.f95'', we create an
empty character array, an instance of the type and its pointer
container and associate the instance to its pointer container:
\begin{verbatim}
  ! empty, allocatable character array
  character, allocatable :: data(:)

  ! declare the instances
  type(StateData),target :: my_state_data
  type(StateDataP) :: my_state_datap
  
  ! associate the pointer
  my_state_dataP%p => my_state_data
\end{verbatim}

Set our state variable:
\begin{verbatim}
  my_state_data%use_igrf = 1
\end{verbatim}

Then marshall the pointer container into the data array:
\begin{verbatim}
  ! marshall the data pointer to our function
  sz = size(transfer(my_state_datap, data))
  allocate(data(sz))
  data = transfer(my_state_dataP, data)
\end{verbatim}

Then, in the function that we've implemented, ``funcPlasmaParams'', we
need to unmarshall this data in order to use it:

\begin{verbatim}
  ! Create the derived type pointer container
  type(StateDataP) :: datap

  ! Unmarshall the callback data
  datap = transfer(funcPlasmaParamsData, datap)

  ! Now you can access the data as datap%p%use_igrf
\end{verbatim}

Finally, run the raytracer, passing our function and state variables
as an argument:
\begin{verbatim}
  call raytracer_run( &
       pos, time, vprel, vgrel, n, &
       B0, qs, ms, Ns, nus, stopcond, &
       pos0, dir0, w, dt0, dtmax, maxerr, maxsteps, root, tmax, &
       fixedstep, del, funcPlasmaParams, data, raytracer_stopconditions)
\end{verbatim}


\end{document}

